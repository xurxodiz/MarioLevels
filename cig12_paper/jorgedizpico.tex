\documentclass[conference]{IEEEtran}
% If the IEEEtran.cls has not been installed into the LaTeX system files,
% manually specify the path to it.  e.g.
% \documentclass[conference]{./IEEEtran}

% Add and required packages here
\usepackage{graphicx,times,amsmath, fontenc, epstopdf}

% Correct bad hyphenation here
\hyphenation{op-tical net-works semi-conduc-tor IEEEtran}

% To create the author's affliation portion using \thanks
\IEEEoverridecommandlockouts

\textwidth 178mm
\textheight 239mm
\oddsidemargin -7mm
\evensidemargin -7mm
\topmargin -6mm
\columnsep 5mm

\begin{document}

% Paper title: keep the \ \\ \LARGE\bf in it to leave enough margin.
\title{\ \\ \LARGE\bf Clustering and Finite State Machines \\ for Adaptive Level Generation in Games}

\author{Jorge Diz Pico (jorge.diz@estudiante.uam.es), David Camacho (david.camacho@uam.es)}

% Uncomment out the following line for invited papers
%\specialpapernotice{(Invited Paper)}

% Make the title area
\maketitle

\begin{abstract}
Content generation has been widely used in gaming
to offer virtually infinite replayability.
Its combination with adaptive techniques also provides a tighter fit between
results and gameplay.
We propose a new approach for procedural level generation
based on a combination of clustering and finite state machines.
The method has been designed and deployed as an entry for the 
Level Generation track of the Mario AI Championship.
We'll show that the implementing architecture provides
great flexibility for both refinement and
extension to other games, and the
experimental results confirm the validity of this approach. 

\end{abstract}

% No keywords

\section{Introduction}
I have such a deep knowledge of what's been made before!

\section{Methodology}
I methoded the shit out of this!

\subsection*{Clustering}
I love weka so much!

\subsection*{Finite State Machines}
I named my cat Turing!

\section{Results}
I got so many interesting things out of this!

\section{Conclusions}
This is great work and I rock!

\subsection*{Future work}
There's so many ways to build upon this!

% Use \section* for the acknowledgments
\section*{Acknowledgments}
I am so thankful!
The authors would like to thank Mr. XYZ for his/her help.
This work was supported in part by the National Science Foundation
under grant no. XXXXX, etc.

% Trigger a \newpage just before a given reference number in order to
% balance the columns on the last page.  Adjust the value as needed;
% it may need to be readjusted if the document is modified later.
%\IEEEtriggeratref{8}
% The "triggered" command can be changed if desired:
%\IEEEtriggercmd{\enlargethispage{-5in}}

% The references section can either be generated by hand or by an
% automatic tool like BibTeX.  If using BibTex, use the standard IEEEtran
% bibliography style.
\bibliographystyle{IEEEtran.bst}
%
% The argument to \bibliography is/are the name(s) of your BibTeX file(s)
% that contains string definitions and bibliography database(s).
\bibliography{IEEEabrv,jorgedizpico}
%
% If you generate the bibliography by hand, or if you copy in the
% resultant .bbl file, set the second argument of \begin to the number of
% references in the bibliography (used to reserve space for the reference
% number labels box).

%\begin{thebibliography}{3}
%\bibitem{book}
%A.~Great, \emph{This is the book title}.\hskip 1em plus 0.5em minus 0.4em\relax
%  This is the name of the publisher, 2006.

%\bibitem{conf}
%F.~Author, S.~Author, and T.~NonRelatedAuthor, ``This is the paper title,'' in
%  \emph{This is the proceedings title}, 2008, pp. 1--8.

%\bibitem{article}
%B.~Myself, ``This is the title of the journal article,'' \emph{This is the name
%  of the journal}, pp. 1--30, 2007.
%\end{thebibliography}

% That's all folks...
\end{document}
