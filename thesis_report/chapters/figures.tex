%%%%%%%%%%%%%%%%%%%%%%%%%%%%%%%%%%%%%%%%%%%%%%%%%%%%%%%%%%%%%%%%%%%%%%%%
\chapter{Manejo de figuras}
%%%%%%%%%%%%%%%%%%%%%%%%%%%%%%%%%%%%%%%%%%%%%%%%%%%%%%%%%%%%%%%%%%%%%%%%

En este capítulo se dan unas nociones bísicas para el manejo de figuras en Latex.

%%%%%%%%%%%%%%%%%%%%%%%%%%%%%%%%%%%%%%%%%%%%%%%%%%%%%%%%%%%%%%%%%%%%%%%%
\section{Címo introducir figuras y referenciarlas en el texto.}
%%%%%%%%%%%%%%%%%%%%%%%%%%%%%%%%%%%%%%%%%%%%%%%%%%%%%%%%%%%%%%%%%%%%%%%%

Así introduzco una figura:

\begin{figure}[H]
    \centerline{\includegraphics[width=5cm]{\figuras/logoudc.eps}}
    \label{fig:mifigura}\caption{Este sería el pie de figura.}
\end{figure}

Y ahora la referencio en el texto diciendo que la Figura \ref{fig:mifigura} es muy bonita.

%%%%%%%%%%%%%%%%%%%%%%%%%%%%%%%%%%%%%%%%%%%%%%%%%%%%%%%%%%%%%%%%%%%%%%%%
\section{Manejo de subfiguras.}
%%%%%%%%%%%%%%%%%%%%%%%%%%%%%%%%%%%%%%%%%%%%%%%%%%%%%%%%%%%%%%%%%%%%%%%%

Ahora incluyo otra figura pero que contiene varias subfiguras dentro de ella. Ademís se puede referencia toda la figura, por ejemplo figura \ref{fig:subfiguras} o bien alguna subfigura en concreto, por ejemplo a las subfiguras \ref{fig:subfig2} y \ref{fig:subfig3}.

%
\begin{figure}[H]
\begin{center}
    \subfigure[Subtítulo de la primera subfigura] {\label{fig:ssubfig1}
        \includegraphics[width=3.5cm]{\figuras/logoudc.eps}
    }
    \hspace{1em} % Permite meter espacio entre las sufiguras. ítil para separar los subtítulos si quedan muy pegados
                 % Si no se precisa se puede eliminar.
    \subfigure[Subtítulo de la la segunda subfigura] {\label{fig:subfig2}
        \includegraphics[width=3.5cm]{\figuras/logoudc.eps}
    }
    \hspace{1em} % Permite meter espacio entre las sufiguras. ítil para separar los subtítulos si quedan muy pegados
                 % Si no se precisa se puede eliminar.
    \subfigure[Subtítulo de la la tercera subfigura] {\label{fig:subfig3}
        \includegraphics[width=3.5cm]{\figuras/logoudc.eps}
    }
    \caption{Ejemplo de una figura con varias subfiguras.}
    \label{fig:subfiguras}
\end{center}
\end{figure}

Finalmente, tambiín se pueden rotar las imígenes e introducir otros efectos, como por ejemplo un cuadro alrededor, tal y como se muestra en la figura \ref{fig:subfiguras2}.
\begin{figure}[H]
\begin{center}
    \subfigure[Logo rotado 45 grados] {\label{fig:subfig21}
        \includegraphics[angle=45,height=2.5cm]{\figuras/logoudc.eps}
    }
    \hspace{3em} % Permite meter espacio entre las sufiguras. ítil para separar los subtítulos si quedan muy pegados
                 % Si no se precisa se puede eliminar.
    \subfigure[Logo rotado 90 grados] {\label{fig:subfig22}
        \includegraphics[angle=90,height=2.5cm]{\figuras/logoudc.eps}
    }
    \\
    \subfigure[Figura con un cuadro alrededor] {\label{fig:subfig24}
        \fbox{ % El comando \fbox permite crear un cuadro alrededor de la figura
            \includegraphics[width=2.5cm]{\figuras/logoudc.eps}
        }
    }
    \caption{Otro ejemplo de una figura con varias subfiguras y con algunos efectos.}
    \label{fig:subfiguras2}
\end{center}
\end{figure}
