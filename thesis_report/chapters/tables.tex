%%%%%%%%%%%%%%%%%%%%%%%%%%%%%%%%%%%%%%%%%%%%%%%%%%%%%%%%%%%%%%%%%%%%%%%%
\chapter{Manejo de Tablas}
%%%%%%%%%%%%%%%%%%%%%%%%%%%%%%%%%%%%%%%%%%%%%%%%%%%%%%%%%%%%%%%%%%%%%%%%

En este capítulo se dan unas nociones bísicas para el manejo de tablas en Latex.

%%%%%%%%%%%%%%%%%%%%%%%%%%%%%%%%%%%%%%%%%%%%%%%%%%%%%%%%%%%%%%%%%%%%%%%%
\section{Címo introducir tablas y referenciarlas en el texto.}
%%%%%%%%%%%%%%%%%%%%%%%%%%%%%%%%%%%%%%%%%%%%%%%%%%%%%%%%%%%%%%%%%%%%%%%%

Así introduzco una tabla:

\begin{table}[H]
  \centering
  \begin{tabular}{ccc}
  %contenido de una tabla de 3x3
    \textbf{columna1} & \textbf{columna2} & \textbf{columna1}\\\hline\hline
    a & b & c \\
    1 & 2 & 3 \\
    x & y & z \\\hline
  \end{tabular}
  \caption{Este es el pie de la tabla}
  \label{tab:mitabla}
\end{table}

Y ahora la referencio en el texto diciendo que la Tabla \ref{tab:mitabla} es muy bonita.

Ahora usamos otra tabla en la que se usa el pie de tabla encima de ella, se agrupan varias columnas con el comando multicolumn y ademís se emplean diferentes grosores de línea.

\begin{table}[H]
\begin{center}
    \caption{Este es el título de otra tabla}
    \begin{tabular}{cccc}
        \multicolumn{4}{c}{\textbf{Ciudad}}\\
        \noalign{\hrule height4pt}
        Madrid & Barcelona & Zaragoza & Tarragona\\
        \noalign{\hrule height2pt}
        Sevilla & Valencia & Santander & Bilbao\\
        \noalign{\hrule height1.5pt}
        A Coruía & Vigo & Santiago & Pontevedra\\
        \noalign{\hrule height1pt}
        Ferrol & Orense & Lugo & Tui\\
        \hline
    \end{tabular}
    \label{tab:mitabla2}
\end{center}
\end{table}

La tabla \ref{tab:mitabla3} contiene el otro ejemplo de una tabla:

\begin{table}[H]
\begin{center}
    \begin{tabular}{|l|c|c|c|c|}
        \hline
         & \multicolumn{2}{|c|}{\textbf{Sistema 1}} & \multicolumn{2}{|c|}{\textbf{Sistema 2}}\\
        \hline
         & Rendimiento & Tiempo CPU & Rendimiento & Tiempo CPU\\
        \hline
        \hline
        Medida 1 & 85 & 11 & 93 & 15\\
        Medida 2 & 90 & 21 & 88 & 20\\
        Medida 3 & 70 & 19 & 87 & 14\\
        Medida 4 & 50 & 20 & 91 & 12\\
        \hline
    \end{tabular}
    \caption{Este es el título de la íltima tabla}
    \label{tab:mitabla3}
\end{center}
\end{table}

%%%%%%%%%%%%%%%%%%%%%%%%%%%%%%%%%%%%%%%%%%%%%%%%%%%%%%%%%%%%%%%%%%%%%%%%
\section{Manejo de tablas de gran longitud.}
%%%%%%%%%%%%%%%%%%%%%%%%%%%%%%%%%%%%%%%%%%%%%%%%%%%%%%%%%%%%%%%%%%%%%%%%

Finalmente, tambiín se pueden incluir tablas que ocupen mís de una pígina como sucede en la tabla \ref{tab:mitabla4}.

\begin{longtable}{|c|c|}
    \caption{Universidades píblicas presenciales.}
    \scriptsize
    \label{tab:mitabla4}\\

    \hline
    \textbf{Comunidad Autínoma} & \textbf{Nombre} \\ \hline \hline
    \endfirsthead
    \multicolumn{2}{c}{\scriptsize\slshape viene de la pígina anterior} \\ \hline
    \textbf{Comunidad Autínoma} & \textbf{Nombre} \\ \hline \hline
    \endhead
    \multicolumn{2}{c}{\scriptsize\slshape continía en la pígina siguiente}\\
    \endfoot
    \hline
    \multicolumn{2}{l}{\scriptsize{FUENTE: Ministerio de Educaciín}}
    \endlastfoot

    \multirow{9}{2cm}{Andalucía} & Almería \\
        & Cídiz \\
        & Círdoba \\
        & Granada \\
        & Huelva \\
        & Jaín \\
        & Mílaga \\
        & Pablo de Olavide \\
        & Sevilla \\\hline
    Aragín & Zaragoza \\\hline
    Asturias & Oviedo \\\hline
    \multirow{2}{2cm}{Canarias} & La Laguna \\
        & Las Palmas de Gran Canaria \\\hline
    Cantabria & Cantabria \\\hline
    Castilla La Mancha & Castilla La Mancha \\\hline
    \multirow{4}{2cm}{Castilla y Leín} & Burgos \\
        & Leín \\
        & Salamanca \\
        & Valladolid \\\hline
    \multirow{4}{2cm}{Cataluía} & Autínoma de Barcelona \\
        & Barcelona \\
        & Girona \\
        & Lleida \\
        & Politícnica de Catalunya \\
        & Pompeu Fabra \\
        & Rovira i Vigill \\\hline
    \multirow{5}{2cm}{Comunidad Valenciana} & Alicante \\
        & Jaume I \\
        & Miguel Herníndez \\
        & Politícnica \\
        & Valencia Estudi General\\\hline
    Extremadura & Extremadura \\\hline
    \multirow{3}{2cm}{Galicia} & A Coruía \\
        & Santiago de Compostela \\
        & Vigo \\\hline
    Islas Baleares & Illes Balears \\\hline
    La Rioja & La Rioja \\\hline
    \multirow{6}{2cm}{Madrid} & Alcalí de Henares \\
        & Autínoma \\
        & Carlos III \\
        & Complutense\\
        & Politícnica \\
        & Rey Juan Carlos \\\hline
    \multirow{2}{2cm}{Murcia} & Murcia \\
        & Politícnica de Cartagena \\\hline
    Navarra & Píblica de Navarra \\\hline
    País Vasco & País Vasco/Euskal Herriko Unibertsitatea  \\
\end{longtable} 