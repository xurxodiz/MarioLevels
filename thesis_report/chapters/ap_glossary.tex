%%%%%%%%%%%%%%%%%%%%%%%%%%%%%%%%%%%%%%%%%%%%%%%%%%%%%%%%%%%%%%%%%%%%%%%%
\chapter{Glossary}
%%%%%%%%%%%%%%%%%%%%%%%%%%%%%%%%%%%%%%%%%%%%%%%%%%%%%%%%%%%%%%%%%%%%%%%%

To create a glossary of terms you need to use package makeidx and then call commands makeindex and printindex.

To mark each term in the text, employ the index scope.

Finally, document must be compiled by the MakeIndex application that Latex brings with itself.


%%%%%%%%%%%%%%%%%%%%%%%%%%%%%%%%%%%%%%%%%%%%%%%%%%%%%%%%%%%%%%%%%%%%%%%%
\section{Glossary, in itself}
%%%%%%%%%%%%%%%%%%%%%%%%%%%%%%%%%%%%%%%%%%%%%%%%%%%%%%%%%%%%%%%%%%%%%%%%

It's probably a good idea to consult on the internet how to make this kind of stuff.

Index will supposedly be created wherever the printindex command is called.

\index{hardware}, diagonal matrix\index{diagonal!matrix}, identity index\index{identity!matrix}.