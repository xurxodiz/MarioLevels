%%%%%%%%%%%%%%%%%%%%%%%%%%%%%%%%%%%%%%%%%%%%%%%%%%%%%%%%%%%%%%%%%%%%%%%%
\chapter{Objectives}
%%%%%%%%%%%%%%%%%%%%%%%%%%%%%%%%%%%%%%%%%%%%%%%%%%%%%%%%%%%%%%%%%%%%%%%%

so we positioned before the situation for games

and some of the challenges its industry faces nowadays

how do they fare against it?

let's go for a ride

%%%%%%%%%%%%%%%%%%%%%%%%%%%%%%%%%%%%%%%%%%%%%%%%%%%%%%%%%%%%%%%%%%%%%%%%
\section{State of the Art}
%%%%%%%%%%%%%%%%%%%%%%%%%%%%%%%%%%%%%%%%%%%%%%%%%%%%%%%%%%%%%%%%%%%%%%%%

What we intend to do.

as we said before

we kinda need replayability and personalization

there are many ways to do this

let's go over them

\subsection{Procedural Content Generation}

replayability

to offer depth

it greatly depends on the genre, of course

we can do it offering different paths

this is cool for example on roleplaying

solving a quest in different ways

damn i should be able to name an example

or graphic adventures sometimes

even fps, if they ask you to destroy a certain thing

that would be changing the gameplay, in a way

changing what you can do

another slightly different option

for games with a straight goal

like racing, or platforms, or puzzles

is to change the level itself

offer a great number of racetracks, or stages, or an order of tetris pieces or pacman levels

it has an advantage

usually in a role game

a role game has to define all possible options to be expanded

you have to realize and implement that maybe the guard can let you pass with a bribe

instead of killing him

but levels

can easily be defined with some rules

and then let an algorithm run wild within those constraints

this approach can be taken as well, perhaps defining personalities for each guard

but it's difficult

and it's mostly unexplored territory so far

this technique, for defining levels with an algorithm, is englobed in what is called

procedural content generation

not only levels, also enemies

the environment, rather

it has a long history

it started with rogue, probably

then let's cite some examples

well i think the aforementioned tetris, no? (check)

oh yeah diablo

minecraft of course


\subsection{Adaptive content}

personalization

like eternal darkness does


fable

oh sims too

and spore


but that's really simple

just changing the content according to a selection

it's not so far

from changing the game

according to difficulty level

adaptive means more

it's the system being smart

and getting to know

who's at the helm

and, itself, adapting to him

l4d2 does it great

there are not many examples more, to be honest...

well

internet matchings work like that

on xbox live for example

making pairings with people on your level

but the technology is pretty much on its infancy still

%%%%%%%%%%%%%%%%%%%%%%%%%%%%%%%%%%%%%%%%%%%%%%%%%%%%%%%%%%%%%%%%%%%%%%%%
\section{Proposal}
%%%%%%%%%%%%%%%%%%%%%%%%%%%%%%%%%%%%%%%%%%%%%%%%%%%%%%%%%%%%%%%%%%%%%%%%

okay so what is the objective of this thesis then

enter the mario ai championship

we have the infinite mario, from notch

on it people experiment with new ai techniques on four different tracks

turing test

agent completion

blah

level generation

we are aiming for the last one

the conditions of the championship are as follows

length of 320x15

player will play a test level

input file will be generated

fed to algorithm

and go!

then they will be evaluated by comparison

we will make a system to enter that competition

our proposal will use clustering to identify the user profile

then, we'll use some grammars that define level prototypes

they have randomness built-in

we will combine them according to the clustering

to create a level

that is, the adaptive generation is taken care of by the clustering

that guides the combination efforts

and the procedural content generation

is taken care by the randomization inherent on the grammars
