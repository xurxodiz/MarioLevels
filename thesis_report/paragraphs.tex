Specially in the first years of the industry, disk space was severely limited. A game's length was primarilly constrained by its size in bytes. In the 1980s, when home computing started to rise [CITE], the main shipping mediums were floppy disks (1.44MB) or casettes (X MB), and home computers had main hard drives in the range of X MB. That put a cap on the amount of different playing levels a game could contain. But the design effort also played a big part. A game that wanted to ship with many different levels needed each of them carefully crafted and balanced by the gamemakers, taking a big chunk of the budget because of the time and money involved. The game industry was still in its infancy, with many of the titles being developed by only one person [CITE], and couldn't afford all that investment.

To solve this problem, early games started to develop algorithms to create levels in a procedural way. That is, creating a set of rules to build game levels on the fly, with a random component, ensuring  giving birth to procedural content generation. The most classic example if Rogue (COMPANY, YEAR), a roleplaying adventure game where the dungeons are generated at the beginning of each game. The influence of its style spawned a new genre of games called roguelike [CITE].